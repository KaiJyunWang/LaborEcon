\documentclass[12pt]{article}

%%%%%%%%%%%%%%%%%%%%%%%%%%%%%%%%%%%%%%%%%%%%%%%%%%%%%%%%%%%%%%%%%%
%%%Comment out to use biblatex instead of bibtex
%%%%%%%%%%%%%%%%%%%%%%%%%%%%%%%%%%%%%%%%%%%%%%%%%%%%%%%%%%%%%%%%%%
\def\UseBibLatex{1}

%%%%%%%%%%%%%%%%%%%%%%%%%%%%%%%%%%%%%%%%%%%%%%%%%%%%%%%%%%%%%%%%%
%Put all your private style files / class style files in 
%the styles / subdirectory. The following command 
%guarantee that latex would find it.
%%%%%%%%%%%%%%%%%%%%%%%%%%%%%%%%%%%%%%%%%%%%%%%%%%%%%%%%%%%%%%%%%

\makeatletter
\def\input@path{{styles/}}
\makeatother


%%%%%%%%%%%%%%%%%%%%%%%%%%%%%%%%%%%%%%%%%%%%%%%%%%%%%%%%%%%%%%%%%%
%A modified usepackge command that checks for style files in 
%the styles / subdirectory.
%%%%%%%%%%%%%%%%%%%%%%%%%%%%%%%%%%%%%%%%%%%%%%%%%%%%%%%%%%%%%%%%%%
\newcommand{\UsePackage}[1]{%
\IfFileExists{styles/#1.sty}{%
\usepackage{styles/#1}%
}{%
\IfFileExists{../styles/#1.sty}{%
\usepackage{../styles/#1}%
}{%
\usepackage{#1}%
}%
}%
}

\usepackage{palatino}
\usepackage[T1]{fontenc}
\usepackage{mathpazo}
\usepackage{textcomp}

\usepackage{setspace}
\usepackage{amsmath}%
\usepackage{amssymb}%
\usepackage{mathtools}
\usepackage{dsfont}
\usepackage[table]{xcolor}%

\setlength{\marginparwidth}{6cm}
\usepackage{todonotes}
\usepackage[in]{fullpage}%
%\usepackage{enumitem}

\usepackage[amsmath,thmmarks]{ntheorem}%
\theoremseparator{.}%

\usepackage{titlesec}%
\titlelabel{\thetitle.\hspace{0.5em}}%
\usepackage{xcolor}%
\usepackage{mleftright}%
\usepackage{xspace}%
\usepackage{graphicx}
\usepackage{hyperref}%
\usepackage{etoolbox}
\usepackage{lipsum}
\usepackage{appendix}
\usepackage{tikz}
\usepackage{mathrsfs}




\newcommand{\hrefb}[3][black]{\href{#2}{\color{#1}{#3}}}%
\newcommand{\fakesubsection}[1]{%
  \par\refstepcounter{subsection}% Increase subsection counter
  \subsectionmark{#1}% Add subsection mark (header)
  \addcontentsline{toc}{subsection}{\protect\numberline{\thesubsection}#1}% Add subsection to ToC
  % Add more content here, if needed.
}

\usepackage{tocloft}
\renewcommand{\cftsecfont}{\bfseries}
\renewcommand{\cftsubsecfont}{\bfseries}
\renewcommand{\cftsecpagefont}{\bfseries}
\renewcommand{\cftsubsecpagefont}{\bfseries}

\setlength{\cftbeforesecskip}{0.5em} % Adjust the space before each section entry
\setlength{\cftbeforepartskip}{2em}  % Adjust the space before each part entry

\usepackage{hyperref}%
\hypersetup{%
unicode,
breaklinks,%
colorlinks=true,%
urlcolor=blue,%
linkcolor=[rgb]{0.5,0.0,0.0},%
citecolor=[rgb]{0,0.2,0.445},%
filecolor=[rgb]{0,0,0.4},
anchorcolor=[rgb]={0.0,0.1,0.2}%
}
\usepackage[ocgcolorlinks]{ocgx2}


%%%%%%%%%%%%%%%%%%%%%%%%%%%%%%%%%%%%%%%%%%%%%%%%%%%%%%%%%%%%%%%%%%%%%%%%
%Defining theorem like environments
%

\theoremseparator{.}%

\theoremstyle{break}%
\newtheorem{theorem}{Theorem}[section]
\theoremstyle{break}%
\newtheorem{lemma}[theorem]{Lemma}
\theoremstyle{break}%
\newtheorem{conjecture}[theorem]{Conjecture}
\theoremstyle{break}%
\newtheorem{corollary}[theorem]{Corollary}
\theoremstyle{break}%
\newtheorem{claim}[theorem]{Claim}%
\theoremstyle{break}%
\newtheorem{fact}[theorem]{Fact}
\theoremstyle{break}%
\newtheorem{observation}[theorem]{Observation}
\theoremstyle{break}%
\newtheorem{invariant}[theorem]{Invariant}
\theoremstyle{break}%
\newtheorem{question}[theorem]{Question}
\theoremstyle{break}%
\newtheorem{proposition}[theorem]{Proposition}
\theoremstyle{break}%
\newtheorem{openproblem}[theorem]{OpenProblem}

\theoremstyle{break}%
\newtheorem*{remark}[theorem]{Remark}%
\theoremstyle{break}%
\newtheorem*{definition}{Definition}
\theoremstyle{break}%
\newtheorem{example}[theorem]{Example}
\theoremstyle{break}%
\newtheorem{exercise}{Exercise}[section]
\theoremstyle{break}%
\newtheorem{problem}{Problem}[section]
\theoremstyle{break}%
\newtheorem{xca}[theorem]{Exercise}
\theoremstyle{break}%
\newtheorem{exercise_h}[theorem]{Exercise}
\theoremstyle{break}%
\newtheorem{assumption}{Assumption}[section]%
\theoremstyle{break}%
\newtheorem{axiom*}{Axiom}%

%Proof environment
\newcommand{\myqedsymbol}{\rule{2mm}{2mm}}

\theoremheaderfont{\itshape}%
\theorembodyfont{\upshape}%
\theoremstyle{nonumberplain}%
\theoremseparator{}%
\theoremsymbol{\myqedsymbol}%
\newtheorem{proof}{Proof.}%
\newenvironment{pf}{%
\begin{proof}$ $\par\nobreak\ignorespaces
}{%
\end{proof}
}

\newtheorem{proofof}{Proofof\!}%

%theorem block end
%%%%%%%%%%%%%%%%%%%%%%%%%%%%%%%%%%%%%%%%%%%%%%%%%%%%%%%%%%%%%%%%%%%%


%%%%%%%%%%%%%%%%%%%%%%%%%%%%%%%%%%%%%%%%%%%%%%%%%%%%%%%%%%%%%%%%%%5
%Color emph

\providecommand{\emphind}[1]{}%
\renewcommand{\emphind}[1]{\emph{#1}\index{#1}}

\definecolor{blue25emph}{rgb}{0,0,11}

\providecommand{\emphic}[2]{}
\renewcommand{\emphic}[2]{\textcolor{blue25emph}{%
\textbf{\emph{#1}}}\index{#2}}

\providecommand{\emphi}[1]{}%
\renewcommand{\emphi}[1]{\emphic{#1}{#1}}

\definecolor{almostblack}{rgb}{0,0,0.3}

\providecommand{\emphw}[1]{}%
\renewcommand{\emphw}[1]{{\textcolor{almostblack}{\emph{#1}}}}%

\providecommand{\emphOnly}[1]{}%
\renewcommand{\emphOnly}[1]{\emph{\textcolor{blue25}{\textbf{#1}}}}

%%%%%%%%%%%%%%%%%%%%%%%%%%%%%%%%%%%%%%%%%

%%%%%%%%%%%%%%%%%%%%%%%%%%%%%%%%%%%%%%%%%%%%%%%%%%%%%%%%%%%%%%%%%%%%%%
%Handling references
%%%%%%%%%%%%%%%%%%%%%%%%%%%%%%%%%%%%%%%%%%%%%%%%%%%%%%%%%%%%%%%%%%%%%%

\newcommand{\HLink}[2]{\hyperref[#2]{#1~\ref*{#2}}}
\newcommand{\HLinkSuffix}[3]{\hyperref[#2]{#1\ref*{#2}{#3}}}

\newcommand{\figlab}[1]{\label{fig:#1}}
\newcommand{\figref}[1]{\HLink{Figure}{fig:#1}}

\newcommand{\thmlab}[1]{{\label{theo:#1}}}
\newcommand{\thmref}[1]{\HLink{Theorem}{theo:#1}}

\newcommand{\exelab}[1]{{\label{exe:#1}}}
\newcommand{\exeref}[1]{\HLink{Exercise}{exe:#1}}

\newcommand{\prolab}[1]{{\label{pro:#1}}}
\newcommand{\proref}[1]{\HLink{Problem}{pro:#1}}

\newcommand{\remlab}[1]{\label{rem:#1}}
\newcommand{\remref}[1]{\HLink{Remark}{rem:#1}}%

\newcommand{\corlab}[1]{\label{cor:#1}}
\newcommand{\corref}[1]{\HLink{Corollary}{cor:#1}}%

\providecommand{\deflab}[1]{}
\renewcommand{\deflab}[1]{\label{def:#1}}
\newcommand{\defref}[1]{\HLink{Definition}{def:#1}}

\newcommand{\lemlab}[1]{\label{lemma:#1}}
\newcommand{\lemref}[1]{\HLink{Lemma}{lemma:#1}}%

\providecommand{\eqlab}[1]{}%
\renewcommand{\eqlab}[1]{\label{equation:#1}}
\renewcommand{\eqref}[1]{\HLinkSuffix{Eq.~(}{equation:#1}{)}}

%%%%%%%%%%%%%%%%%%%%%%%%%%%%%%%%%%%%%%%%%%%%%%%%%%%%%%%%%%%%%%%%%%%

\newcommand{\remove}[1]{}%


\renewcommand{\th}{th\xspace}




%%%%%%%%%%%%%%%%%%%%%%%%%%%%%%%%%%%%%%%%%%%%%%%%%%%%%%%%%%%%%%%%%%%%%%%%%
%Defining comptenum environment using enumitem
\usepackage{enumitem}

\newlist{compactenumA}{enumerate}{5}%
\setlist[compactenumA]{topsep=0pt,itemsep=-1ex,partopsep=1ex,parsep=1ex,%
label=(\Alph*)}%

\newlist{compactenuma}{enumerate}{5}%
\setlist[compactenuma]{topsep=0pt,itemsep=-1ex,partopsep=1ex,parsep=1ex,%
label=(\alph*)}%

\newenvironment{thmenum}{%
\begin{enumerate}[label=\small(\alph*\small),topsep=8pt,itemsep=4pt,partopsep=0pt, parsep=0pt]
}{%
\end{enumerate}
}

\newlist{compactenumI}{enumerate}{5}%
\setlist[compactenumI]{topsep=0pt,itemsep=-1ex,partopsep=1ex,parsep=1ex,%
label=(\Roman*)}%

\newlist{compactenumi}{enumerate}{5}%
\setlist[compactenumi]{topsep=0pt,itemsep=-1ex,partopsep=1ex,parsep=1ex,%
label=(\roman*)}%

\newlist{compactitem}{itemize}{5}%
\setlist[compactitem]{topsep=0pt,itemsep=-1ex,partopsep=1ex,parsep=1ex,%
label=\ensuremath{\bullet}}%


%%%%%%%%%%%%%%%%%%%%%%%%%%%%%%%%%%%%%%%%%%%%%%%%%%%%%%%%%%%%%%%%%%%%%%%%%%

%%%%%%%%%%%%%%%%%%%%%%%%%%%%%%%%%%%%%%%%%%%%%%%%%%%%%%%%%%%%%%%%%%%
%%%%%%%%%%%%%%%%%%%%%%%%%%%%%%%%%%%%%%%%%%%%%%%%%%%%%%%%%%%%%%%%%%%

\numberwithin{figure}{section}%
\numberwithin{table}{section}%
\numberwithin{equation}{section}%



%%%%%%%%%%%%%%%%%%%%%%%%%%%%%%%%%%%%%%%%%%%%%%%%%%%%%%%%%%%%%%%%%%%
%%%%%%%%%%%%%%%%%%%%%%%%%%%%%%%%%%%%%%%%%%%%%%%%%%%%%%%%%%%%%%%%%%%
%Papers specific commands...
%%%%%%%%%%%%%%%%%%%%%%%%%%%%%%%%%%%%%%%%%%%%%%%%%%%%%%%%
%%%%%%%%%%%%%%%%%%%%%%%%%%%%%%%%%%%%%%%%%%%%%%%%%%%%%%%%



%%%%%%%%%%%%%%%%%%%%%%%%%%%%%%%%%%%%%%%%%%%%%%%%%%%%%%%%
%%Begin Ipe Preamble
%%%%%%%%%%%%%%%%%%%%%%%%%%%%%%%%%%%%%%%%%%%%%%%%%%%%%%%%


%%%%%%%%%%%%%%%%%%%%%%%%%%%%%%%%%%%%%%%%%%%%%%%%%%%%%%%%
%%End Ipe Preamble
%%%%%%%%%%%%%%%%%%%%%%%%%%%%%%%%%%%%%%%%%%%%%%%%%%%%%%%%
%
\newcommand{\Set}[2]{\left\{ #1 \;\middle\vert\; #2 \right\}}

\newcommand{\pth}[1]{\mleft(#1\mright)}%

\newcommand{\E}{{\mathbb{E}}}
\newcommand{\diam}[1]{\operatorname{diam}\mleft(#1\mright)}
\newcommand{\osc}[2]{\operatorname{osc}\mleft(#1,\, #2\mright)}

\newcommand{\Ex}[1]{\E\mleft[ #1 \mright]}

\newcommand{\ceil}[1]{\mleft\lceil {#1} \mright\rceil}
\newcommand{\floor}[1]{\mleft\lfloor {#1} \mright\rfloor}
\makeatletter
\renewcommand\d[1]{\mspace{6mu}\mathrm{d}#1\@ifnextchar\d{\mspace{-3mu}}{}}
\makeatother
\def\at{
  \left.
  \vphantom{\int}
  \right|
}

\newcommand{\brc}[1]{\left\{ {#1} \right\}}
\newcommand{\sbrc}[1]{\mleft[#1\mright]}
\newcommand{\set}[1]{\brc{#1}}%

\newcommand{\abs}[1]{\left\lvert {#1} \right\rvert}%
\newcommand{\norm}[1]{\left\lVert {#1} \right\rVert}

\newcommand{\ds}{\displaystyle}%

\newcommand{\R}{\mathbb{R}}%
\newcommand{\N}{\mathbb{N}}
\newcommand{\Q}{\mathbb{Q}}
\newcommand{\Z}{\mathbb{Z}}
\newcommand{\C}{\mathbb{C}}
\newcommand{\F}{\mathcal{F}}
\newcommand{\A}{\mathcal{A}}

\newcommand{\exist}{\exists\:}
\newcommand{\forany}{\forall\:}

\DeclareMathOperator*{\argmax}{arg\,max}


\title{
    An Introduction to Function Iterations for Dynamic Programming 
    }

\author{%
   Kai-Jyun Wang
   \thanks{Department of Economics; National Taiwan University. 
   Email: \url{b11303072@ntu.edu.tw}.}
}

\date{\today}

\begin{document}
\setstretch{1.2}

\maketitle

\section*{Foreword}
This document aims at providing mathematical details for value 
function iteration (VFI) in economics. For implementation in Julia, 
please refer to Sargent's fantastic 
\href{https://julia.quantecon.org/dynamic_programming/optgrowth.html}
{website}. This document serves as a supplement to the website.

\tableofcontents

\section{Optimal Growth Model}
In this capter, we introduce an optimal growth model. The model is 
going to be our working example for VFI. 

Consider an agent who seeks to maximize his lifetime expected utility. 
The agent's problem is to choose his future path of consumption $c_t$ 
and capital stock $k_{t+1}$, subject to the constraint: 
\begin{equation}\eqlab{constraint}
    c_t + k_{t+1} \leq y_t,
\end{equation} 
where both $c_t$ and $k_{t+1}$ are non-negative. $y_t$ is the agent's 
income at time $t$, which follows the law of motion: 
\begin{equation}\eqlab{production}
    y_t = z_t f(k_t),\quad z_t \overset{iid}{\sim} \phi,
\end{equation}
where $z_t$ is a random variable that follows a positively supported 
distribution $\phi$. $f(k_t)$ is the production function. 

\begin{assumption}
    The production function $f(k_t)$ is continuous and increasing in 
    $k_t$.
\end{assumption}

The agent's optimization problem is given by: 
\begin{equation}
    v(y_t) = \max_{c_t} \E_0\sbrc{\sum_{t=0}^{\infty} \beta^t u(c_t)},
\end{equation}
subject to the constraints \eqref{constraint} and \eqref{production},
where $\beta \in (0, 1)$ is the discount factor, and $u(c_t)$ is the 
utility flow in each period. $v(y_t)$ is called the \textbf{value 
function} and $y_t$ is called the \textbf{state variable} of $v$. We 
further take two assumptions on $u(\cdot)$ and $v(\cdot)$.

\begin{assumption}
    The utility function $u(c_t)$ is continuous and increasing in 
    $c_t$.
\end{assumption}

Note that by this assumption, the inequality in \eqref{constraint} 
is replaced by an equality since if $c_t + k_{t+1}$ is strictly less 
than $y_t$, the agent can always increase $c_t$ to improve the 
utility. 

\begin{assumption}
    The value function $v(y_t)$ is bounded.
\end{assumption}

Note that we may also write value function as follows. 
\begin{equation}
    \begin{split}
        v(y_0) 
        &= \max_{c_t} \E_0\sbrc{\sum_{t=0}^{\infty} \beta^t u(c_t)}\\
        &= \max_{c_t} \E_0\sbrc{u(c_0) + \beta\sum_{t=1}^{\infty} \beta^{t-1} u(c_t)}\\
        &= \max_{c_t} u(c_0) + \beta\E_0\sbrc{\sum_{t=0}^{\infty} \beta^{t} u(c_{t+1})}\\ 
        &= \max_{c_0} u(c_0) + \beta\E_0\sbrc{v(y_1)}\\
        &= \max_{c_0} u(c_0) + \beta\int v(z_1f(y_0-c_0))\phi(dz_1).
    \end{split}
\end{equation}
The form is called the \textbf{Bellman equation}. It is a functional 
equation regarding $v$. Note that the true value function would solve 
this functional equation. The Bellman equation approach 
has a significant advantage compared to the traditional method of 
Lagrange multiplier. The Bellman equation approach transforms an 
infinite horizon problem into a two-period problem, and also deals 
with the uncertainty. However, there is a clear drawback: How to find 
$v$? 

\section{Mathematical Details} 
In this section, we provide mathematical details for solving the 
Bellman equation. For those who do not interest in the details, it 
is safe to skip this section.

We begin by introducing some fundamental concepts in analysis. 

\begin{definition}
    A \textbf{metric space} is a pair $(X, d)$, where $X$ is a set 
    and $d: X \times X \to \R$ is a function that satisfies the 
    following properties: 
    \begin{thmenum}
        \item $d(x, y) \geq 0$ for all $x, y \in X$; $d(x, y) = 0$ 
        if and only if $x = y$.
        \item $d(x, y) = d(y, x)$ for all $x, y \in X$.
        \item $d(x, z) \leq d(x, y) + d(y, z)$ for all $x, y, z \in X$.
    \end{thmenum}
    $d$ is called a \textbf{metric} (\textbf{distance}) on $X$.
\end{definition}

\begin{definition}
    A sequence $\set{x_n}$ in a metric space $(X, d)$ is said to be 
    converge to $x \in X$ if for every $\epsilon > 0$, there exists 
    $N \in \N$ such that $d(x_n, x) < \epsilon$ for all $n \geq N$.
\end{definition}

\begin{definition}
    A sequence $\set{x_n}$ in a metric space $(X, d)$ is said to be 
    \textbf{Cauchy} if for every $\epsilon > 0$, there exists $N \in 
    \N$ such that $d(x_n, x_m) < \epsilon$ for all $n, m \geq N$.
\end{definition}

\begin{definition}
    A metric space $(X, d)$ is said to be \textbf{complete} if every 
    Cauchy sequence in $X$ converges to a point in $X$.
\end{definition}

\begin{remark}
    $\R^n$ is a complete metric space under the Euclidean metric $d(x,y) = 
    \sqrt{\sum_{i=1}^{n}(x_i-y_i)^2}$.
\end{remark}

\begin{definition}
    A normed space $X$ is a vector space with scalar field $\R$
    equipped with a norm $\norm{\cdot}$, satisfying that 
    \begin{thmenum}
        \item $\norm{x} \geq 0$ for all $x\in X$; $\norm{x} = 0$ if 
        and only if $x = 0$.
        \item $\norm{ax} = \abs{a}\norm{x}$ for all $a\in\R$ and $x\in X$.
        \item $\norm{x+y} \leq \norm{x} + \norm{y}$ for all $x, y\in X$.
    \end{thmenum}
\end{definition}

\begin{remark}
    The scalar field $\R$ can be replaced by other fields, but 
    for our purpose, we only consider $\R$.
\end{remark}

\begin{remark}
    The norm induces a metric $d(x, y) = \norm{x-y}$. In fact, 
    the Euclidean norm $\norm{x} = \sqrt{\sum_{i=1}^{n}x_i^2}$ 
    induces the Euclidean metric. For this reason, a normed 
    space is automatically a metric space and the metric is 
    defined by its norm.
\end{remark}

\begin{definition}
    $B(X)$ is the set of all real-valued bounded continuous functions 
    defined on $X$.
\end{definition}

\begin{proposition}
    $B(X)$ is a complete metric space under the supremum norm
    $\ds \norm{f} = \sup_{x \in X} \abs{f(x)}$.
\end{proposition}
\begin{pf}
    Let $\set{f_n}$ be a Cauchy sequence in $B(X)$. For each $x \in 
    X$, define $f(x) = \lim_{n\to\infty} f_n(x)$. The limit exists 
    since $\set{f_n(x)}$ is a Cauchy sequence in $\R$. We claim that
    $f\in B(X)$. First, $f$ is bounded since for each $x \in X$, 
    there exists $N$ such that $\abs{f_n(x) - f_m(x)} < \epsilon$ 
    for all $n, m \geq N$. Letting $m = N$ and $n\to\infty$ yields 
    that $\abs{f(x) - f_N(x)}\leq \epsilon$. Hence, $\abs{f(x)} \leq 
    \abs{f_N(x)} + \epsilon$ for all $x \in X$. Second, $f$ is 
    continuous since for each $x \in X$ and $\epsilon > 0$, we 
    may pick $\delta > 0$ such that $\abs{f_N(x) - f_N(y)} < 
    \epsilon$ for all $y \in X$ with $d(x, y) < \delta$. Hence, 
    $\abs{f(x) - f(y)} \leq \abs{f(x) - f_N(x)} + \abs{f_N(x) - 
    f_N(y)} + \abs{f_N(y) - f(y)} < 3\epsilon$ for all $y \in X$ 
    with $d(x, y) < \delta$. Since $\epsilon$ is arbitrary, $f$ 
    is indeed continuous and hence $f\in B(X)$. This completes 
    the proof.
\end{pf}

\begin{definition}
    An operator $T: X \to X$ is called a \textbf{contraction} 
    if there exists $\alpha \in (0, 1)$ such that $d(T(x), T(y)) \leq 
    \alpha d(x, y)$ for all $x, y \in X$.
\end{definition}

\begin{theorem}[Contraction Mapping Theorem]
    Let $(X, d)$ be a complete metric space and $T: X \to X$ be a 
    contraction mapping with contraction factor $\alpha \in (0, 1)$. 
    Then $T$ has an unique fixed point $x^* \in X$. That is, $Tx^* 
    = x^*$. Furthermore, for any $x_0\in X$, the sequence $\set{x_n}$ 
    defined by $x_{n+1} = Tx_n$ converges to $x^*$.
\end{theorem}
\begin{pf}
    For each $x_0 \in X$, we define $x_n = T^n(x_0)$. Then 
    \begin{equation}
        d(x_{n+1}, x_n) = d(T^{n+1}(x_0), T^n(x_0)) \leq \alpha^n 
        d(x_1, x_0)\to 0 \quad \text{as } n\to\infty.
    \end{equation}
    Hence, $\set{x_n}$ is a Cauchy sequence. Since $X$ is complete, 
    $\set{x_n}$ converges to some $x^* \in X$. Next, suppose both 
    $x^*$ and $y^*$ are fixed points of $T$. Then 
    \begin{equation}
        d(x^*, y^*) = d(T(x^*), T(y^*)) \leq \alpha d(x^*, y^*) 
        < d(x^*, y^*),
    \end{equation}
    posing a contradiction. Therefore, $x^*$ is unique.  
\end{pf}

\begin{theorem}[Blackwell's Theorem]
    Suppose $T: B(X) \to B(X)$ satisfies the following properties: 
    \begin{thmenum}
        \item $T$ is monotone, i.e., $f \leq g$ implies $Tf \leq Tg$.
        \item There exists $\alpha\in (0,1)$ such that for any 
        $c\in \R_+$, $T(f+a) \leq Tf + \alpha c$.
    \end{thmenum}
    Then $T$ is a contraction.
\end{theorem}
\begin{pf}
    Suppose $f, g \in B(X)$ and $c\in\R_+$ satisfy the conditions 
    (a) and (b). Then 
    \begin{equation}
        f \leq g + \norm{f-g}.
    \end{equation}
    Thus we have 
    \begin{equation}
        Tf \leq T(g + \norm{f-g}) \leq Tg + \alpha\norm{f-g}.
    \end{equation}
    Rearranging the terms yiels the desired result.
\end{pf}

We now turn back to the Bellman equation. 

\begin{definition}
    The \textbf{Bellman operator} $T: v \mapsto Tv$ is defined by 
    \begin{equation}
        Tv(y) = \max_{c} u(y) + \beta\int v(zf(y-c))\phi(dz).
    \end{equation}
\end{definition}
\begin{remark}
    The solution to the Bellman equation is the fixed point of the 
    Bellman operator $T$.
\end{remark}

\begin{corollary}
    The Bellman operator $T$ is a contraction.
\end{corollary}
\begin{pf}
    We are going to check the conditions in Blackwell's Theorem are 
    satisfied. First, if $v\leq w$, $v,w\in B(X)$, let $c_v$ and 
    $c_w$ be the optimal consumption for $v$ and $w$, respectively. 
    Then 
    \begin{equation}
        \begin{split}
            Tv(y) 
            &= u(c_v) + \beta\int v(zf(y-c_v))\phi(dz)\\
            &\leq u(c_v) + \beta\int w(zf(y-c_v))\phi(dz)\\
            &\leq u(c_w) + \beta\int w(zf(y-c_w))\phi(dz) = Tw(y).
        \end{split}
    \end{equation} 
    Thus the monotonicity is satisfied. Second, for any $c\in\R_+$, 
    \begin{equation}
        \begin{split}
            T(v+c)(y) 
            &= u(c) + \beta\int v(zf(y-c))\phi(dz)\\
            &\leq u(c) + \beta\int v(zf(y-c))+c \phi(dz)
            = Tv(y) + \alpha c.
        \end{split}
    \end{equation}
    This completes the proof.
\end{pf}

Since $B(X)$ is a complete metric space and $T$ is a contraction 
operator on it, by the contraction mapping theorem, $T$ has an 
unique fixed point. This fixed point is the solution to the 
Bellman equation. Also, the proof of the contraction mapping 
theorem reveals a numerical algorithm to find the fixed point:
\begin{thmenum}
    \item Start with a guess $v_0\in B(X)$.
    \item Apply the Bellman operator $T$ to $v_0$ to get $v_1 = 
    Tv_0$. 
    \item Compare $v_1$ with $v_0$. If they are close enough, 
    stop; otherwise, set $v_0 = v_1$ and repeat step 2.
\end{thmenum} 
The algorithm is called the \textbf{value function iteration}. 


\end{document}